\documentclass{article}
\usepackage{amsmath}
\usepackage{polyglossia}
\usepackage{fullpage}
\setmainlanguage{russian}
\setotherlanguage{english}
\newfontfamily\russianfont[Script=Cyrillic,Mapping=tex-text]
{CMU Sans Serif}


\begin{document}

\title{Домашняя работа}
\author{Алина Крамар}

\maketitle
\begin{enumerate}

\item[6.8]

У нас есть $X$ покрывающие паросочетание. Разобьем $Y$ на два множества $Y' \cup Y''$.
$Y''$ будет состоять из вершин, которые не являются концами ребер из этого паросочетания.
Очевидно, что все "плохие" ребра из условия задачи не могут заканчиваться в $Y''$,
поэтому по умолчанию считаем, что все $Y = Y'$. Будем кодировать каждое ребро двумя вершинами.
Может ли так получиться, что двумя вершинами мы закодировали и то и другое. Эта ситуация возможна в случае,
когда мы получили пары $(x_1, x_2)$ и $(x_2, x_1)$. Но в таком случае ребра хорошие, т.к. они принадлежат другому
покрывающему паросочетанию. Т.о. порядок тоже не важен, и значит их получается $\binom {|X|}2$.

Почему достигается. Рассмотрим пример, в кототом между долями (если нарисовать вершинки на одной высоте)
 есть параллельные ребра.
Так же из доли $X$ в $Y$ идут ребра по нправлению вниз. Если мы выберем такое ребро в паросочетание,
то не сможем покрыть все вершинки сверху, т.к. из $Y$ туда ребер нет. Таких ребер может быть только
$\binom {|X|}2$
\item[6.9]

Докажем по индукции по $s$.

База: $s = 0$ - критерий Холла.

Допустим, что при $s = k$ все верно. Добавим к доле $Y$ еще одну вершинку,
из которой идут ребра во все вершинки доли $X$. Для этого графа выполняется условие при $(m - s)$.
Когда мы выберем в новом графе паросочетание. Не более чем одно ребро проходит через добавленную вершину,
поэтому когда мы его уберем, полуится $(m - s - 1)$.
\end{enumerate}

\end{document}
