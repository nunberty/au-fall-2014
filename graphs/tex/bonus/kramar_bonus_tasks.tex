\documentclass{article}
\usepackage{amsmath}
\usepackage{amssymb}
\usepackage{amsfonts}
\usepackage{polyglossia}
\usepackage{fullpage}
\setmainlanguage{russian}
\setotherlanguage{english}
\newfontfamily\russianfont[Script=Cyrillic,Mapping=tex-text]
{CMU Sans Serif}


\begin{document}

\title{Задачи для двоечников}
\author{Алина Крамар}

\maketitle
\begin{enumerate}

\item

Докажем по индукции.

Для $n=1$ верно.

Рассмотрим $n-$мерный куб. Он
устроен как два $(n-1)$-мерных куба, соединенных ребрами. По ИП мы знаем, что в
каждом $(n-1)$-мерном кубе есть гамильтонов путь. Он начинается из
какого-то соединаяющего ребра. Т.о дополнив два пути из $(n-1)$-
мерного подграфа этим ребром, получаем гамильтонов путь во всем
графе. Это справедливо для всех ребер мужду двумя $(n-1)$-мерными
кубами. Выбрав пару из них, получим гамильтонв цикл.

\item

Рассмотрим самый длинный путь в графе $x_{1},...,x_{k}$.

Если он гамильтонов, то все
хорошо.

Допустим он не гамильтонов,
значит есть вершина $y$, которая на этом пути не лежит. Если она не лежит
на этом пути, значит из нее есть исходящее ребро в конец пути $\{y \to
x_{k}\}$ и входящее ребро из начала этого пути $\{x_{i} \to y\}$. Во
все вершины этого $x_{2},...,x_{k-1}$ пути из
$y$ есть исходящире ребра, иначе можно было бы
увеличить путь на одну вершину. Но у нас есть ребро $\{x_{i} \to y\}$. Значит пройдя от начала по этому ребру в нележащую на пути
вершину получили путь $x_{1},y,...,x_{k}$, больший по длине. Значит
наш путь был не максимальный. Получили противоречие.

\item

Рассмотрим любой цикл в турнире $\{x_{1},...,x_{k}\}$.
Докажем, что его можно увеличить на
единичку.

Допустим, что в графе есть вершина $z$, из которой и в которую
есть ребра из цикла (обязательно смешанные, т.е. $\{x_{i} \to z\}$ и
$\{z \to x_{j}\}$).  Найдется такая
пара соседних вершин $x_{p}, x_{p+1}$, для которых будут ребра
$\{x_{p} \to z\}$ и $\{z \to x_{p+1}\}$. Проложим цикл
$\{x_{1},...,x_{p},z,x_{p+1},...,x_{k}\}$. Получили цикл с длиной,
большей чем был, на 1.

Если таких вершин нет,
значит из всех остальных вершин $z_{1'},...,z_{(n-k)'}$, не лежащих в
цикле,
ребра имеют вид либо $\{z_{j'} \to x_{i}\}$ $\forall j \in
\{1:(n-k)\}$ $\forall i \in \{1:k\}$, либо $\{x_{i} \to z_{j'}\}$ $\forall j \in
\{1:(n-k)\}$ $\forall i \in \{1:k\}$. Т.к. граф сильносвязен,
то таких вершин как минимум две, и между ними есть какой-то путь. Этот
путь проходит через вершины, котрые либо имеют только исходящие в цикл
ребра, или только входящие. Рассмотрм на этом пути первую смену типа
вершины $z_{q}, z_{q+1}$. В цикле найдутся три такие вершины $y_{j-1},
y_{j}, y_{i+1}$, что бутут ребра $\{z_{q} \to y_{j-1}\}$ и $\{y_{j+1}
\to z_{q+1}\}$. Получаем цикл
$\{x_{1},...,y_{j-1},z_{q},z_{q+1},y_{j+1},...,x_{k}\}$, который
длиньше предыдущего на 1. Т.о. мы можем удлинить любой цикл на
единицую.

Значит есть и гамильтонов цикл (удлинили до $n$).


Теперь докажем, что есть все циклы длины $i$ для $i \in
\{3:n\}$. Любая хорда $\{x_{i} \to x_{j}\}$ укорачивает
цикл $\{x_{1},...,x_{k}\}$. Возьмем либо $x_{i},...,x_{j} \cup
\{x_{j}, x_{i}\}$, либо $x_{j},...,x_{i} \cup
\{x_{i}, x_{j}\}$. Получили цикл заведомо меньшей длины. И т.к. мы
умеем увеличивать лдину цикла на 1, то последвательные уменьшения
хордами, а затем увеличения на 1 дадут циклы длины $i$.

\item -

\item

Рассмотрим вершину $z$ с максимльной исходящей степенью, $deg(z) = m =
\max_{v \in V}\{deg(v)\}$. Рассмотрим все вершины $x_{1},...,x_{m}$, из которых можно за 1 шаг добраться из
вершины $z$. Допустим, что есть вершина $y$, до которой нельзя добраться
за один шаг из эти вершин. Т.к. мы рассматриваем турнир, из $y$ в
$\{x_i | i \in 1..m\}$
ребра направлены, т.к. $y \notin \{x_i | i \in 1..m\}$, то между $y$ и $z$ может
быть только ребро $\{y, z\}$, но тогда $deg(y) = m + 1$, значит у $z$
не максимальная степень. Получили противоречие.

\item

Допустим в графе есть точка сочленения $z$. Рассмотрим получающиеся
половинки $G_1$ и $G_2$. И $G_1$, и $G_2$ -- это двудольный
граф. Допустим в $G_1$ первая доля $X$, а вторая -- $Y \cup z$.

Из $X$ в $Y$ ведут $d|X|$ ребер, из $Y$ в $X$ -- $d|Y| + d'$,
где $d'$ -- количество ребер из вершины $z$, лежащих в $G_1$,
причем $0 < d' < d$.

Получается, что $d(|X| - |Y|) = d'$, $|X| - |Y| \in \mathbb{Z}$ и $d'$
делит $d$, $d' > 0$ и $d' < d$. Получили противоречие.

Значит точек сочленения в графе нет.

\item

Будем доказываеть для тех графов, у которых нет вершин степени 0,
иначе это неправда.

Допустим граф $G$ не двудольный. Значит в нем есть цикл нечетной длины
$x_1,...,x_k$, где $k \in 2\mathbb{N}+1$.
Рассмотрим реберные автоморфизмы. Если есть такие автоморфизмы, в
которых два соседних ребра можно совместить (т.е. ребра эквивалентны) сдвигом, то
две соседние вершины эквивалентны (как концы ребра). В обратном случае любые два ребра
будут эквивалентны в силу нечетности цикла. Значит любые две вершины
эквивалентны, значит граф вершинно-транзитивный, а этого быть не
может.

Значит он двудольный.


\item -

\item -

\item

Допустим два длинейших цикла $x_1,...,x_k$ и $y_1,...,y_p$ не пересекаются вовсе. Т.к. граф связен,
то между ним есть путь $x_i,...,y_j$. Т.к. граф не содержит точек сочленения, то
мужду ними есть еще один путь $x_{i'},...,y_{j'}$, причем он не
пересекается с первым.

Тут сейчас я не знаю, как написать понятно, но могу рассказать устно
или катинку нарисовать.
Два цикла $x_1,...,x_k$ и $y_1,...,y_p$ и два пути $x_{i},...,y_{j}$ и
$x_{i'},...,y_{j'}$ задают 6 кусочков цикла (2 пути, 2 половинки
первого и две половинки второго). Комбинируя эти кусочки, получаем 4
пары циклов, суммарная длина хотя бы одного из которых больше изначально рассматриваемых
циклов. Если суммарная длина больше, значит какой-то из них больше
какого-то из изначальных, значит циклы были не наибольшие. Получили
противоречие.

Допустим, наибольние циклы пересекаются по одной
вершине. Т.к. точки сочлененя нет, то между ними есть еще какой-то
простой путь. По рассуждениям, приведенным выше (считая, что один из
путей длины 0), получаем противоречие.

Значит циклы пересекаются как минимум по двум вершинам.

\end{enumerate}

\end{document}
