\documentclass{article}
\usepackage{amsmath}
\usepackage{polyglossia}
\usepackage{fullpage}
\usepackage{amssymb}
\usepackage{amsfonts}
\setmainlanguage{russian}
\setotherlanguage{english}
\newfontfamily\russianfont[Script=Cyrillic,Mapping=tex-text]
{CMU Sans Serif}


\begin{document}

\title{Домашняя работа}
\author{Алина Крамар}

\maketitle
\begin{enumerate}

\item

Сигнатура $(f^{1}, =^2)$ задана на $\mathbb{Z}$, $[f](x) = x + 2$.

Предикат $y = x + 1$ невыразим, потому что неустойчив относительно
автоморфизма

$$
g =
\left\{
  \begin{array}{ll}
    x     & \mbox{if } x  \vdots 2  \\
    x + 2 & otherwise
  \end{array}
\right.
$$

\item

Сигнатура $(=^2, p^2)$, с носителем $\mathbb{N}_+$ и $[p](x, y) = x$
$|$ $y$.

Предикат $x = 2$ невыразим, потому что неустойчив относительно
автоморфизма $f$, который будет использовать запись любого
натурального числа в виде произведения простых по основной теореме
арифметики
$n=p_1^{\alpha_{1}}p_2^{\alpha_2}...p_k^{\alpha_k}$. Обозначим степень
при $p_i = P$ как $\alpha(P)$. Таким образом степень у двойки равна
$\alpha(2)$.

Автоморфим $f$ меняет степени у двойки и у тройки, если они есть в
представлении числа. Если нет, то степень равна 0.
Это отображение переводит двойку в тройку и наоборот.

\item logic1.png из письма
\item logic2.png
\item logic3.png
\item logic4.png
\item logic5.png
\item logic6.png
\item logic7.png
\end{enumerate}

\end{document}
