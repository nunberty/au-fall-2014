\documentclass{article}
\usepackage{amsmath}
\usepackage{polyglossia}
\usepackage{comment}
\usepackage{fullpage}
\setmainlanguage{russian}
\setotherlanguage{english}
\newfontfamily\russianfont[Script=Cyrillic,Mapping=tex-text]
{CMU Sans Serif}


\begin{document}

\title{Домашняя работа 2}
\author{Алина Крамар}

\maketitle
\begin{enumerate}

\item
  \begin{itemize}
  \item $\neg p \land \neg q \land \neg r \lor \neg p \land \neg q
    \land r \lor \neg p \land q \land \neg r \lor \neg p \land q
    \land r \lor p \land q \land \neg r$

    $$\neg p \land \neg q \land \neg r \lor \neg p \land \neg q
    \land r \lor \neg p \land q \land \neg r \lor \neg p \land q
    \land r \lor p \land q \land \neg r =$$ $$\neg p \land \neg q \lor
    \neg p \land \neg r \lor \neg p \land r \lor \neg p \land r \lor q
    \land \neg r =$$ $$\neg p \lor q \land \neg r$$

\begin{tabular}{|c|c|c|c|c|c|}
% BEGIN RECEIVE ORGTBL foobar
\hline
 & $\neg{}$ p $\land$ $\neg{}$ q $\land$ $\neg{}$ r & $\neg{}$ p $\land$ $\neg{}$ q $\land$ r & $\neg{}$ p $\land$ q $\land$ $\neg{}$ r & $\neg{}$ p $\land$ q $\land$ r & $\land$ r $\lor$ p $\land$ q $\land$ $\neg{}$ r \\
\hline
$\neg{}$ p & x & x & x & x &  \\
\hline
q $\land$ $\neg{}$ r &  &  & x &  & x \\
\hline
% END RECEIVE ORGTBL foobar
\end{tabular}

\begin{comment}
#+ORGTBL: SEND foobar orgtbl-to-latex :splice t :skip 0
  |----------------+----------------------------------+-----------------------------+-----------------------------+------------------------+-------------------------------------|
  |                | \neg p \land \neg q \land \neg r | \neg p \land \neg q \land r | \neg p \land q \land \neg r | \neg p \land q \land r | \land r \lor p \land q \land \neg r |
  |----------------+----------------------------------+-----------------------------+-----------------------------+------------------------+-------------------------------------|
  | \neg p         | x                                | x                           | x                           | x                      |                                     |
  |----------------+----------------------------------+-----------------------------+-----------------------------+------------------------+-------------------------------------|
  | q \land \neg r |                                  |                             | x                           |                        | x                                   |
  |----------------+----------------------------------+-----------------------------+-----------------------------+------------------------+-------------------------------------|

\end{comment}


\item $\overline{pq}rs+\overline{p}qrs+p\overline{q}rs+pq\overline{rs}+pq\overline{r}s+pqr\overline{s}+pqrs$

  $$\overline{pq}rs+\overline{p}qrs+p\overline{q}rs+pq\overline{rs}+pq\overline{r}s+pqr\overline{s}+pqrs
  =$$
$$\overline{p}rs+\overline{q}rs+qrs+prs+pq\overline{r}+pq\overline{s}+pqs+pqr
=$$ $$pq+rs$$


\begin{tabular}{|l|l|l|l|l|l|l|l|}
% BEGIN RECEIVE ORGTBL table2
\hline
 & $\overline{pq}rs$ & $\overline{p}qrs$ & $p\overline{q}rs$ & $pq\overline{rs}$ & $pq\overline{r}s$ & $pqr\overline{s}$ & $pqrs$ \\
\hline
$pq$ &  &  &  & x & x & x & x \\
\hline
$rs$ & x & x & x &  &  &  & x \\
\hline
% END RECEIVE ORGTBL table2
\end{tabular}

\begin{comment}
#+ORGTBL: SEND table2 orgtbl-to-latex :splice t :skip 0
|------+-------------------+-------------------+-------------------+-------------------+-------------------+-------------------+--------|
|      | $\overline{pq}rs$ | $\overline{p}qrs$ | $p\overline{q}rs$ | $pq\overline{rs}$ | $pq\overline{r}s$ | $pqr\overline{s}$ | $pqrs$ |
|------+-------------------+-------------------+-------------------+-------------------+-------------------+-------------------+--------|
| $pq$ |                   |                   |                   | x                 | x                 | x                 | x      |
|------+-------------------+-------------------+-------------------+-------------------+-------------------+-------------------+--------|
| $rs$ | x                 | x                 | x                 |                   |                   |                   | x      |
|------+-------------------+-------------------+-------------------+-------------------+-------------------+-------------------+--------|
\end{comment}

\end{itemize}

\item $p\overline{q}+\overline{p}q\overline{r}$

$$p(q \oplus 1) \oplus (p \oplus 1)q(r \oplus 1) \oplus
p\overline{p}q\overline{qr} = $$
$$pq \oplus p \oplus pq \oplus q \oplus qr \oplus pqr = $$
$$p \oplus q \oplus qr \oplus pqr $$


\item Построить полином Жегалкина для функции с заданой таблицей истинности


\begin{tabular}{|r|r|r|c|}
% BEGIN RECEIVE ORGTBL table3
\hline
$x$ & $y$ & $z$ & $f(x, y, z)$ \\
\hline
0 & 0 & 0 & 1 \\
\hline
0 & 0 & 1 & 0 \\
\hline
0 & 1 & 0 & 0 \\
\hline
0 & 1 & 1 & 0 \\
\hline
1 & 0 & 0 & 1 \\
\hline
1 & 0 & 1 & 0 \\
\hline
1 & 1 & 0 & 0 \\
\hline
1 & 1 & 1 & 1 \\
\hline
% END RECEIVE ORGTBL table3
\end{tabular}
\begin{comment}
#+ORGTBL: SEND table3 orgtbl-to-latex :splice t :skip 0
|-----+-----+-----+--------------|
| $x$ | $y$ | $z$ | $f(x, y, z)$ |
|-----+-----+-----+--------------|
|   0 |   0 |   0 |            1 |
|-----+-----+-----+--------------|
|   0 |   0 |   1 |            0 |
|-----+-----+-----+--------------|
|   0 |   1 |   0 |            0 |
|-----+-----+-----+--------------|
|   0 |   1 |   1 |            0 |
|-----+-----+-----+--------------|
|   1 |   0 |   0 |            1 |
|-----+-----+-----+--------------|
|   1 |   0 |   1 |            0 |
|-----+-----+-----+--------------|
|   1 |   1 |   0 |            0 |
|-----+-----+-----+--------------|
|   1 |   1 |   1 |            1 |
|-----+-----+-----+--------------|
\end{comment}

$$f = a_{0} \oplus a_{1}x \oplus a_{2}y \oplus a_{3}z \oplus
a_{12}xy \oplus a_{13}xz \oplus a_{23}yz \oplus a_{123}xyz $$

$f(0, 0, 0) = a_{0} = 1$

$f(0, 0, 1) = 1 \oplus a_{3} = 0 \Rightarrow a_{3} = 1$

$f(0, 1, 0) = 1 \oplus a_{2} = 0 \Rightarrow a_{2} = 1$

$f(0, 1, 1) = 1 \oplus 1 \oplus 1 \oplus a_{23} = 0 \Rightarrow a_{23}
= 1$

$f(1, 0, 0) = 1 \oplus a_{1} = 1 \Rightarrow a_{1} = 0$

$f(1, 0, 1) = 1 \oplus 1 \oplus a_{13} = 0 \Rightarrow a_{13} = 0$

$f(1, 1, 0) = 1 \oplus 1 \oplus a_{12} = 0 \Rightarrow a_{12} = 0$

$f(1, 1, 1) = 1 \oplus 1 \oplus 1 \oplus 1 \oplus a_{123} = 1
\Rightarrow a_{123} = 1$

Ответ: $f(x,y,z) = 1 \oplus y \oplus z \oplus yz \oplus xyz$

\item Докажите неполноту системы связок:

  \begin{enumerate}
  \item $(\oplus, \land)$

    $\oplus$ и $\land$ из $T_{0}$, значит не выполнен критерий Поста.

  \item $(\land, \lor, \to)$

    Тут тоже обе из $T_{o}$, соответственно критерий Поста так же не выполнен.
  \item $(\neg)$

    Отрицание не является линейной функцией (в 1 равна 0). Т.о. не
    выполняется критерий Поста.
  \end{enumerate}

\item Докажите полноту системы связок
  \begin{enumerate}
    \item $(1, \lor, \oplus)$

      Выразим через эти связки другую полную систему связок. Отрицание
      выражается как $x \oplus 1$. В свою очередь конъюнкция по
      законам Моргана выражается как $(x \oplus 1) \lor (y \oplus 1)
      \oplus 1$. Может быть может короче и красивее, но мне в голову
      не пришло.
    \item $(\to, 0)$

      Выразим через эти связки другую полную систему связок. Мы знаем,
      что система связок $(\lor, \neg)$ полна. Поэтому выразим эти
      операции через $\to$ и $0$.

      $(x \to 0) \to y$ - дизъюнкция.

      $x \to 0$ - отрицание.

      В принципе этого достаточно, т.к. конъюнкция выражается через
      отрицание и дизъюнкцию по законам Моргана.
  \end{enumerate}
\end{enumerate}

\end{document}
