\documentclass{article}
\usepackage{amsmath}
\usepackage{polyglossia}
\usepackage{fullpage}
\setmainlanguage{russian}
\setotherlanguage{english}
\newfontfamily\russianfont[Script=Cyrillic,Mapping=tex-text]
{CMU Sans Serif}


\begin{document}

\title{Домашняя работа 1}
\author{Алина Крамар}

\maketitle
\begin{enumerate}

\item
  \begin{enumerate}
    \item $\land$
      \begin{itemize}
      \item $\neg , \lor $

        $x \land y \leftrightarrow \neg \neg (x \land y) \leftrightarrow \neg
        (\neg x \lor \neg y)$
      \item $\to , \bot $

        $(x \to (y \to \bot)) \to \bot $
      \item $\downarrow$

        $(x \downarrow x) \downarrow (y \downarrow y)$
      \end{itemize}

    \item $\lor$
      \begin{itemize}
      \item $\to , \bot $

        $(x \to \bot) \to y$
      \item $\downarrow$

        $(x \downarrow y) \downarrow (x \downarrow y)$
      \end{itemize}

    \item $\to$
      \begin{itemize}
      \item $\neg , \lor$

        $x \to y \leftrightarrow \neg x \lor y$
      \item $\downarrow$

        $((x \downarrow x) \downarrow y) \downarrow ((x \downarrow x) \downarrow y)$
      \end{itemize}

    \item $\neg$
      \begin{itemize}
      \item $\to , \bot $

        $x \to \bot$
      \item $\downarrow$

        $(x \downarrow x)$
      \end{itemize}

    \item $\bot$
      \begin{itemize}
      \item $\neg , \lor $

        $\neg (x \lor \neg x)$
      \item $\downarrow$

        $(x \downarrow x) \downarrow ((x \downarrow x) \downarrow (x \downarrow x))$
      \end{itemize}

    \item $\top$
      \begin{itemize}
      \item $\neg , \lor $

        $x \lor \neg x$
      \item $\to , \bot $

        $\bot -> x$
      \item $\downarrow$

        $(x \downarrow (x \downarrow x)) \downarrow (x \downarrow (x \downarrow x))$
      \end{itemize}

  \end{enumerate}

\item
  \begin{enumerate}
    \item $(a \to b) \leftrightarrow \neg a \lor b \leftrightarrow b
      \lor \neg a \leftrightarrow \neg b \to \neg a $

    \item $((((a \to b) \to a) \to a) \to b) \to b \leftrightarrow
      (\top \to b) \to b \leftrightarrow b \to b \leftrightarrow \top$
      - по закону Пирса

    \item

      $(a \to (b \land c)) \leftrightarrow ((a \to b)\land(a \to c))$

      $(\neg a \lor (b \land c)) \leftrightarrow ((\neg a \lor
      b)\land(\neg a \lor c))$

      $(\neg a \lor (b \land c)) \leftrightarrow (\neg a \lor (b \land
      c))$ - по дистрибутивности

    \item $(a \lor b \to c) \leftrightarrow ((a \to c) \lor (b \to
      c))$

      Вообще тут не тавтология при $a = \top, b = c = \bot$. Видимо,
      там перепутана птичка во втором. НО даже если и перепутали, то
      доказывается как в предыдущем пункте.
    \item по задаче 4
  \end{enumerate}

\item
  \begin{enumerate}
    \item $(( b \to c) \lor b) \land a \to c$

      $c = \top, a = \bot$
    \item $(((a \land b) \to c) \to c) \lor \neg a$

      $a = \top, b = c = \bot$

    \item $((a \leftrightarrow b) \leftrightarrow b) \leftrightarrow
      (c \leftrightarrow a)$ - по задаче 4
  \end{enumerate}

\item

  Заметим, что связка эквивалентности коммутативна и
  ассоциативна. Т.о. путем эквивалентных логических преобразований мы
  можем добиться такой записи формулы, где все одинаковые переменные будут
  стоять подряд. Выражение вида $x \leftrightarrow x$ заменяется на
  тождественную истину. В свою очередь выражение вида $x \leftrightarrow
  \top$ эквивалентно просто $x$. Сокращая таким образом выражение,
  получаем, что либо все одинаковые переменные сократились (было
  четное число), либо остались
  в единственном экземпляре (в случае нечетного количества). Не трудно
  увидеть, что последний случай не является тавтологией, т.к. оставшиеся
  переменные могут иметь любую оценку.
\end{enumerate}

\end{document}
