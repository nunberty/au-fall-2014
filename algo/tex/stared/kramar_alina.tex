\documentclass{article}
\usepackage{amsmath}
\usepackage{polyglossia}
\usepackage{fullpage}
\setmainlanguage{russian}
\setotherlanguage{english}
\newfontfamily\russianfont[Script=Cyrillic,Mapping=tex-text]
{CMU Sans Serif}


\begin{document}

\title{Задачи со звездочкой}
\author{Алина Крамар}

\maketitle
\begin{enumerate}

\item[7.6*]

Раскрасим граф как-нибудь. Затем будем исправлять раскраску.
Будем итерироваться по вершинам, пока не получим нужную раскраску. При
этом на каждой итерации будем смотреть, если у вершины больше чем один
сосед такого же цвета, то будем инвертировать цвет вершины. При этом
на каждой итерации количество разноцветных ребер будет
уменьшаться. Кроме того, т.к. степень вершины не привосходит трех, то
таких переборов будт не много. Поэтому сложность $O(|V|)$
сохраняется. (Я говорю $O(|V|)$, потому что $|E| < 3|V|$).

\end{enumerate}

\end{document}
