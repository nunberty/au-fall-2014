\documentclass{article}
\usepackage{amsmath}
\usepackage{polyglossia}
\usepackage{fullpage}
\setmainlanguage{russian}
\setotherlanguage{english}
\newfontfamily\russianfont[Script=Cyrillic,Mapping=tex-text]
{CMU Sans Serif}


\begin{document}

\title{Домашняя работа 10}
\author{Алина Крамар}

\maketitle
\begin{enumerate}

\item[1.]

Остовное дерево может быть не единственным, если в графе присутствуют
ребра с одинаковым весом. Проще будет понять на алгоритме
Краскала. Деревья могут получиться разными, потому что в случае ребер
с одинаковым весом в алгоритме при выборе ребра появляется
завуалированный `произвол`.

Итак: модифицируем алгоритм Краскала.
Сортируем все ребра. Т.к. мы рассматриваем ребра последовательно, то
от порядка, который нам вернула сортировка, зависит, какое ребро будет
рассмотрено первым. Помечаем ребра, имеющие один вес в бакет. Дальше
итерируемся не по ребрам, а по бакетам ребер. (с точки зрения
реализации -- границы бакета, это элементы, которые отличаюся весом от
своего соседа).

Рассматриваем все ребра бакета. Если добавление ребра уменьшает
количество компонент связности, то добавляем его в отдельную структуру, которая хранит все потенциальные
ребра, уменьшающие своим добавлением компоненты, если не уменьшает
число -- никуда не добавляем, а пропускаем ребро. Затем проходим по
полученному множеству и добавляем все ребра, если их добавление
уменьшает количество компонент связности в получающемся графе после
добавления предыдущих. Если найдутся ребра, которые нельзя добавить,
значит остовное дерево не единственно и оно зависит от порядка добавления,
если же коллекция осталась пуста, значит и при рассмотрении Краскалом
без модификации все ребра оказались бы в остове, значит дерево
единственно. Сложность соответствует заданной, т.к. на каждой итерации
мы просматриваем все ребра, которые и так бы просмотрели и в
дополнение потом еще раз проходим по коллекции, что увеличивает не
более чем на логарифм, и зависит от того, как организовать структуру.

\item[2.]

Чтобы доказать, что предложенный в задаче алгоритм строит минимальное
остовное дерево, рассмотрим разницу между полученным в нем деревом
($MST_{1}$) и деревом, полученным в алгоритме Краскала ($MST_{2}$). Стягивание ребер по сути является
объединением в компоненту связности, поэтому на каждом шаге алгортма
из задачи мы рассматриваем соединяющие компонены связности ребра, как
и в алгоритме Краскала.

Перенумеруем ребра, которые мы добавили в алгоритме и в алгоритме
Краскала в порядке добавления к минимальному остову. Дойдем до первой
разницы, например, ребро $e_{k}$.  Добавим это ребро к
$MST_{1}$. Т.к. это дерево, то новое ребро добавит цикл. Рассмотрим
этот цикл. В нем все ребра весят не менее, чем $e_k$, потому что мы
рассматривам минимальные по весу ребра, выходящие из вершины, и он
точно не меньше, чем минимально по весу ребро во всем графе на данном
шаге, потому что все меньшие мы уже просмотрели и они
совпадают. Т.о. мы можем заменить любое ребро на наше $e_k$, что
приведет к неувеличению суммарного веса. Но такой ситуации не может
быть, потому что во втором алгоритме мы рассмаривали наименьшие,
выходящие из вершины ребра, и если мы его находили, то добавляли, не
рассматривая дальше аналогичные ребра. Т.о. мы бы добавили меньшее
ребро раньше, чем добавили бы ребро с
большим весом. Следовательно, ребро такого же веса. Получается, что удаление
любого ребра из цикла лишь задаст другое дерево, но такое бывает только
в случае кратных ребер. Т.о. алгоритмы строят одинаковые деревья с
точностью до порядка рассмотрения алгоритмом Краскала.

\item[3.]
При уменьшении веса одного из ребер (допустим $\{x, y\}$) графа $G$, рассмотрим
$MST(G) \cup \{x, y\}$. В полученном графе есть один цикл, и этот цикл
содержит измененное ребро. Пройдя по этому циклу, удалим из него
максимальное ребро, удаление которого не нарушает связность и
разрывает цикл. Максимальное кол-во ребер, которое мы можем
просмотреть равно кол-ву вершин (в том случае, кода минимальный остов
представлял собой линию, а добавленное ребро соединяло его концы). Это
работает, потому что в минимальном остовном дереве содержатся
минимальные допустимые ребра (как, например, строит Краскал),
соединяющие данные компоненты связности и части графа. Поэтому цикл
возникнет в том месте, где выбор может измениться, но компоненты тем
не менее остаются связными, а на другие части графа это не влияет.
Поэтому сложность будет линейная от количества вершин.
\item[4*.]

\begin{enumerate}
\item

В полном графе вторым приближением дерева Штайнера является минимальное
остовное дерево построенное на вершинах из T. Если строить его
алгоритмом Прима, то выполняется заданая сложность. Теперь объясним,
почему это так. Рассмотри разницу между оптимальным деревом Штайнера и
полученнм остовом. Удвоим все ребра в дереве Штайнера и обойдем его по
Эйлеровскому циклу. Получили в точности два дерева Штайнера, но т.к. в
графе выполняется неравенство треугольника, то это точно больше, чем
построенный на этих вершинах остов.

\item

Запускаем алгоритм Краскала. Дожидаемся, когда добавим ребро,
объединяющее все терминалы в одну компоненту связности. Затем опускаем
все пути, концами которых не являются терминалы. Как это
делается. начала опускаем все ребра, которые ``торчат'' из полученного
графа. Так же стоит заметить, что деревом Штайнера будет являться
какой-то путь с отростками по ходу его движения. Но т.к. у нас нет
циклов, то уменьшить эти пути уже не получится. Это будет
2-приближением, потому что удвоенное оптимальное дерево будет больше,
чем минимальный остов, построенный на суженом кол-ве вершин.

\end{enumerate}

\end{enumerate}

\end{document}
